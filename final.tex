% Created 2022-05-09 Mon 09:42
% Intended LaTeX compiler: pdflatex
\RequirePackage{rotating}
\documentclass[12pt]{amsart}
\usepackage[utf8]{inputenc}
\usepackage{lmodern}
\usepackage{graphicx}
\usepackage{longtable}
\usepackage{float}
\usepackage{wrapfig}
\usepackage{rotating}
\usepackage[normalem]{ulem}
\usepackage{textcomp}
\usepackage{marvosym}
\usepackage{wasysym}
\usepackage{hyperref}
\usepackage{minted}
\usepackage{booktabs}
\usepackage[authordate-trad,backend=biber,natbib]{biblatex-chicago}
\usepackage{wasysym}
\newcommand{\Cov}{\ensuremath{\mbox{Cov}}}
\renewcommand{\Pr}{\ensuremath{\mbox{Pr}}}
\newcommand{\Eq}[1]{(\ref{eq:#1})}
\usepackage{bm}\usepackage{econometrics}
\newcommand{\T}{\top}
\newtheorem{proposition}{Proposition} \newcommand{\Prop}[1]{Proposition \ref{prop:#1}}
%\newtheorem{problem}{Problem} \newcommand{\Prob}[1]{Problem \ref{prob:#1}}
%\newtheorem{theorem}{Theorem} \newcommand{\Thm}[1]{Theorem \ref{thm:#1}}
%\newtheorem{corollary}{Corollary} \newcommand{\Cor}[1]{Corollary \ref{cor:#1}}
%\newtheorem{remark}{Remark} \newcommand{\Rem}[1]{Remark \ref{rem:#1}}
%\newtheorem{condition}{Condition} \newcommand{\Cond}[1]{Condition \ref{cond:#1}}
%\newtheorem{lemma}{Lemma} \newcommand{\Lem}[1]{Lemma \ref{lem:#1}}
%\newtheorem{assumption}{Assumption} \newcommand{\Ass}[1]{Assumption \ref{ass:#1}}
\newcommand{\Fig}[1]{Figure \ref{fig:#1}} \newcommand{\Tab}[1]{Table \ref{tab:#1}}
\usepackage{dsfont}\newcommand{\one}{\ensuremath{\mathds{1}}}
\usepackage{xcolor}
\newcommand{\rv}[1]{\ensuremath{\textcolor{red}{#1}{}}}
%\newcommand{\rv}[1]{\ensuremath{{}_{rv}{#1}{}}}
%\newcommand{\rv}[1]{\ensuremath{\underline{#1}{}}}
\newcommand{\rvy}{\rv{y}}
\newcommand{\rvX}{\rv{X}}
\newcommand{\rvx}{\rv{x}}
\newcommand{\rvu}{\rv{u}}
\renewcommand{\do}[1]{\ensuremath{\mbox{do}(#1)}}
\renewcommand{\E}{\ensuremath{\mathds{E}}}
\usepackage{fullpage}
\renewcommand{\thesection}{\Roman{section}}
\DeclareMathOperator*{\argmax}{arg\,max}
\DeclareMathOperator*{\argmin}{arg\,min}
\addbibresource{main.bib}
\author{Ethan Ligon}
\date{\today}
\title{ARE212 Final Exam}
\hypersetup{
 pdfauthor={Ethan Ligon},
 pdftitle={ARE212 Final Exam},
 pdfkeywords={},
 pdfsubject={},
 pdfcreator={Emacs 28.0.50 (Org mode 9.5.2)}, 
 pdflang={English}}
\begin{document}

\maketitle

This is the final exam for ARE212, covering material from the second
half of the course taught in Spring 2022.   The exam is
``take-at-home''; you may consult any resources you wish in completing
it (notes, textbooks, lecture videos, etc.) except for other
people.  This last restriction isn't easily enforceable; I rely on
you to approach this as principled adults who adhere to the
Berkeley Honor Code.

More guidance:
\begin{itemize}
\item The exam is due at 10am  on Tuesday May 10.
\item In completing the exam you should develop written arguments
(e.g., expressed using \LaTeX{} or pencil and paper).  In some cases
you may wish to supplement these written arguments with
computation, such as Monte Carlo experiments.  Should you do so,
please provide me with your working, open source, well-documented code.  (This
last could be links to a github repo, a Jupyter notebook attached
to an email, or similar).  In any case please be sure that
materials you submit are well-organized and clearly
documented---if I overlook some file you've sent or can't run it
that's on you.
\item You are welcome (and even encouraged) to use arguments developed
in our \texttt{piazza} discussions, but in this case please clearly cite
the person and discussion (e.g., ``As argued by Michelle in a
discussion `Tests of Normality' (\texttt{@53\_f11}) the optimal weighting
matrix can be written as a function of the unknown parameters.'')
\item Please email files or links to \texttt{ligon@berkeley.edu}.
\end{itemize}

\section{Some short questions}
\label{sec:org4757ca0}
\begin{enumerate}
\item The model of Wright (1934) is ``normal'' in the sense that all of
the underlying random variation comes from two normally
distributed random variables.  Why are there ``only five useful
moments''? Why aren’t higher order moments involved?  If there
were three normally distributed random variables how many ``useful
moments'' would there be?

\item From a sample of \(N\) original observations \((y_i,X_i)\) we
estimate the vector of parameters \(\beta\) in
\(\vy = \mX\beta + \vu\), using ordinary least squares, obtaining
\(\beta^{(OLS)}\).  

It's often good practice to use some sort of resampling procedure
to estimate the finite sample distribution of our estimators.

Consider two different approaches to this:
\begin{description}
\item[{Bootstrap}] From a sample of \(N\) original observations \((y_i,X_i)\) we
 randomly select \(N\) observations (with replacement), and for
 selection \(s\) we estimate the vector of parameters \(\beta\) in
 \(\vy^{(s)} = \mX^{(s)}\beta + \vu^{(s)}\), obtaining
 \(\beta^{(s)}\).  We do this \(N\) times (for some not very
well-motivated reason), so that we wind up with \(N\) different
estimates of \(\beta\).
\item[{Cross-Validation}] From the same sample of \(N\) original observations \((y_i,X_i)\) we
 compute \(N\) different ``leave-out-one'' estimates of \(\beta\) from
 \(\vy^{(-i)} = \mX^{(-i)}\beta + \vu^{(-i)}\), each time obtaining
 \(\beta^{(-i)}\), based on \(N-1\) observations.  This again gives
us \(N\) different estimates of \(\beta\).
\end{description}

How could you use the Bootstrap or Cross-Validation sets of
estimates to construct estimates of the standard errors of the
OLS estimator \(\beta^{(OLS)}\)?  What  relationship might you
expect between the sample variance of the Bootstrap estimates and
the cross-validation estimates?

\item We wish to estimate the model \(\rvy = \rvX\beta + \rvu\) using a
sample of \(N\) original, \emph{independent} observations \((y_i,X_i)\).
The economic model we're interested in implies \(\E(\rvu|\rvX)=0\) so that
we have \(\E(\mX^\T\rvu|\rvX=\mX) = 0\), but says
nothing about the variance of \(\rvu\) across observations.
\begin{enumerate}
\item Use the identifying assumption to derive the GMM estimator of
\(\beta\).
\item Using the fact that the observations are independent, what can
you say about the structure of the covariance matrix \(\E
        \mX^\T\rvu\rvu^\T\mX\)?
\item Use your result in (b) to construct an estimator of the
covariance matrix of your estimator \(\beta\).  How does this
compare with the estimator in the homoskedastic case?
\end{enumerate}
\end{enumerate}



\section{General Weighted Regressions}
\label{sec:orgebc91b6}
Consider the regression \(\rvy = \rvX\beta + \rvu\), where \(\E\rvu\rvu^\T=\Omega\).
We've discussed a variety of estimators of \(\beta\) under different
conditions, and asserted that many of these can be expressed as the
solution to a set of linear equations written in matrix form as
\begin{equation}
   \mT^\T \vy = \mT^\T \mX\beta,
\end{equation}
where \((\mX,\vy)\) are observed data.
For each of the following estimators, what is \(\mT\), and under what
conditions will the estimator be (i) consistent; and (ii)
asymptotically efficient?
\begin{enumerate}
\item OLS
\item Two-stage least squares
\item Generalized least squares
\item Generalized Method of Moments
\item If you replace the matrix \(\mT\) with the matrix \emph{function}
\(\mT(x)\),  and replace the linear regression with the
non-parametric form \(\rvy=m(\rvx) + \rvu\), then the kernel regression
(Nadaraya-Watson) estimator can be similarly expressed in the form
\(\mT(x)\vy = \mT(x)m(x)\).  What is \(\mT(x)\) in this case?
\end{enumerate}

\section{Clustered sample}
\label{sec:orga33187f}
We wish to estimate a relationship \(\rvy = \rvX\beta + \rvu\), and
adopt the identifying assumption that \(\E(\rv{Z}^\T\rvu) = 0\), with
\(\E\rv{Z}^\T\rvX=\mQ\) having full column rank (and more rows than
columns).

\begin{enumerate}
\item With an sample of \(N\) observations \((X_i,y_i,Z_i)\), use the
identifying assumption and the assumption that \(\E
     \rvu\rvu^\T=\Omega\) (with \(\Omega\) unknown to the econometrician)
and construct an asymptotically efficient GMM estimator of
\(\beta\).  Describe its limiting distribution.
\item As in the previous question, but assume \(\Omega=\sigma^2\mI\)
(still unobserved).  Compare with the Two-stage least-squares
estimator.  Compare the efficiency of this estimator with the
estimator in (1).  Explain.
\item Now suppose that the sample is based on a clustered-design, so
that each of the \(N\) observations is drawn from one of \(K\leq
     N/2\) clusters.  We assume \(\E\rvu_i=\sigma^2\) for all
observations.  But for any two distinct observations within a
cluster we assume \(\E \rvu_i\rvu_j = \gamma\), while for two
observations \emph{across} clusters we assume \(\E \rvu_i\rvu_j = 0\).
How could you exploit this structure in a GMM estimator?  Compare
the asymptotic variance of this estimator with that in (1) and
(2).
\end{enumerate}

\section{Uniform Random Variables}
\label{sec:orgc20a2e4}
Suppose a scalar random variable \(\rvx\) is uniformly distributed on
the interval \([a,b]\).  Then the \(n\)th moment of \(\rvx\) is given
  \[
  \E(\rvx^{n})=\frac {b^{n+1}-a^{n+1}}{(n+1)(b-a)}.
  \]
\begin{enumerate}
\item Suppose we have a sample of \(N\) realizations of \rvx,
\(\{x_i\}_{i=1}^N\).  Suggest a just identified estimator of the
unknown parameters \(a\) and \(b\) based on the expression for the
moments of \rvx.
\item What can you say about the relationship of the smallest value of
\(x_i\) in your sample relative to your estimate of \(a\)?
\item Now suppose that we are interested in testing the \emph{hypothesis}
that \(\rvx\) is uniformly distributed.   Suggest a practical test
based again on the moments of \(\rvx\) under the null hypothesis of uniformity.
\end{enumerate}

\section{Omitted Variables}
\label{sec:org2df7c01}
You are asked to serve as a referee for a paper submitted to a top
field journal.  In the submitted paper the researcher uses a sample
of size \(N\) to estimate a model
\[
     y = \alpha + \beta x + u.
  \]
The coefficient \(\beta\) seems to be significantly different from
zero, but the researcher is concerned about omitted variable bias,
so they also estimate a variety of alternative specifications of the form
\[
     y = \alpha + \beta x + \gamma w + u,
  \]
where \(w\) is one of a number of other variables that the researcher
hypothesizes might have some effect on \(y\) as a way of testing the
first model.

The researcher finds a particular variable \(w\) which enters the
regression significantly, and so (i) rejects the first model,
concluding that the first estimate of \(\beta\) was in fact affected by
omitted variable bias; (ii) declares the augmented regression to be
their ``preferred specification;'' and (iii) proceeds to construct
standard \(t\)-statistics for \(\beta\) and \(\gamma\) as a way of
proceeding with inference.

Peer reviews in economics usually include some ``notes for the
author.''  What might your notes say about the paper's approach to
omitted variable bias?  Comment specifically on each of (i), (ii),
and (iii).  Try to make your remarks critical yet
constructive---what shortcomings do you see, and how might the
author address these?

\section{Nested Samples}
\label{sec:orgdf1ab2a}
Consider the linear model \(y=X\beta + u\), where \(X\) is thought to
depend on \(u\), but where we have a set of instruments \(Z\) such that
\(\E Z^\T u = 0\).  In this case our observations on \(y\) are limited,
in that we don't always observe \(y\) even when we do observe
\((X,Z)\).  We can think of this as having two samples, nested in the
following way.  We have \(N_1\) iid observations on the triple
\((y,X,Z)_1\) but \(N_2>N_1\) iid observations on \((X,Z)_2\), with \((X,Z)_1\)
(i.e., the observations on \(X\) and \(Z\) in the first dataset) a
subset of \((X,Z)_2\).  How can we best make use of all these data?
\begin{enumerate}
\item One econometrician suggests an augmented sort of two-stage-least
squares approach, using the richer dataset to estimate a
linear relationship \(X_2 = Z_2\pi + v\), and thus constructing a
``first-stage'' prediction equation \(\hat{X} = Z\hat{\pi}_2\) which
is more precisely estimated that it would be in the usual case in
which only data in \((y,X,Z)_1\) was exploited.
\begin{enumerate}
\item Continue the argument by substituting into the second stage.
What can you say about the properties of the augmented
estimator compared to the properties of the usual
two-stage least squares estimator?
\item Under what conditions would the augmented estimator be
preferred to two-stage least squares on just the sample of
\(N_1\) observations?
\end{enumerate}

\item A second econometrician suggests using the smaller sample to
construct a sample moment condition \((Z_1^\T y_1)=(Z_1^\T X_1)b\),
and argues that if \(b\) in this condition identifies \(\beta\), then
it should be possible to construct \(\hat{u}_2 = \hat{y}_2 -
     X_2b\), and that for this larger set of observations we must have
\(\E Z_2^\T\hat{u}_2 = 0\).  She argues that these two sets of moment
conditions could then be combined into an over-identified
optimally-weighted GMM estimator.
\begin{enumerate}
\item How would you construct the optimal GMM weighting matrix for
this approach?  Derive an expression for the asymptotic
variance matrix for the estimator \(b\).  How does it depend on
the larger sample?
\item Comment on this approach.  Does the second set of moment
conditions add useful information?
\item If you also knew that \(u\) was homoskedastic how could you
exploit this information?  How would the resulting estimator
compare with two-stage-least squares?  What can you say about
the relative efficiency of this estimator versus two-stage
least squares?
\end{enumerate}
\end{enumerate}


\printbibliography
\end{document}